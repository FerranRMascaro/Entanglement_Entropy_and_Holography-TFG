\documentclass[lettersize,journal]{IEEEtran}

\usepackage[utf8]{inputenc}
\usepackage{lmodern}
\usepackage[T1]{fontenc}
\usepackage{mathtools}
\usepackage{amsmath}
\usepackage{float}
\usepackage{csvsimple}
\providecommand{\abs}[1]{\lvert#1\rvert}
\usepackage{braket}
\providecommand{\eq}[2]{
    \begin{equation}
        #2
    \label{eq:#1}
    \end{equation}
}
\usepackage{amsmath}
\DeclareMathOperator{\calA}{\mathcal{A}}
\DeclareMathOperator{\calB}{\mathcal{B}}
\DeclareMathOperator{\calH}{\mathcal{H}}
\DeclareMathOperator{\calL}{\mathcal{L}}
\DeclareMathOperator{\tr}{tr}
\usepackage{fancyhdr}
\usepackage{authblk}
\usepackage{abstract}
\usepackage{wrapfig}
\usepackage{graphicx}
\usepackage{hyperref}

\usepackage[backend=biber,style=phys]{biblatex}
\addbibresource{TFG.bib}

% \usepackage{multirow}
% \usepackage[table,xcdraw]{xcolor}

% \usepackage{graphicx}
% \usepackage{caption}
% \usepackage{subcaption}

% \usepackage[a4paper]{geometry}
% \geometry{top=3cm, bottom=3.3cm, left=3cm, right=3cm}


\title{Entanglement Entropy and Holography - Examples}
\author{F. R. Mascaró}
\date{}

\begin{document}


\maketitle{}


% \twocolumn[
%     \begin{@twocolumnfalse}
%         \maketitle{}
%         \begin{abstract}
%         ...
%         \end{abstract}
%     \end{@twocolumnfalse}
% ]


\section{Examples of EE}

The general procedure is explained in the main paper.


\subsection{EE of a disk in $d=3$}

A circuilar region $\calA$ with radius $R$ inside a 2-dimensional layer $z = \delta \ll 1$ is represented in polar coordinates as
\eq{1A}{
    \calA = \{ ( r, \theta, z, t ) \ | \ t = 0, z = \delta, r \le R \} \ ,
}
and the corresponding surface of minimal area $\delta_{\calA}$ will be represented by the $z$ coordinate as a function of the polar coordinates of $\calA$:
\eq{1gammaA}{
    \gamma_{\calA} = \{ ( r, \theta, z, t ) \ | \ t = 0, z = f (r, \theta) \} \ .
}
There is no property on the AdS spacetime theory that will make the symmetry of $\partial \calA$ on the coordinate $\theta$ not being transfered to $\gamma_{\calA}$. Thus, $z = f (r)$.

In polar coordinates, the metric corresponding to a ($d+1$)-dimensional AdS spacetime will be
\eq{1Ametric}{
    ds^2_{\text{AdS}_4} = g_{\mu \nu} dx^\mu dx^\nu = 
    \frac{L^2}{z^2} [ -dt^2 + dz^2 + dr^2 + r^2 d\theta^2 ] \ .
}

Applying the restrictions on the coordinates for the definition of $\gamma_{\calA}$, it is obtained the induced metric of the surface:
\eq{1gammaAmetric}{
    ds^2_{\gamma_{\calA}} = h_{\rho \sigma} dx^\rho dx^\sigma = 
    \frac{L^2}{f(r)^2} \left[ \left( 1+ \dot{f}(r)^2 \right) dr^2 + r^2 d\theta^2 \right] \ ,
}
being $dz = \frac{\partial z}{\partial x^\rho} dx^\rho = \frac{\partial f(r)}{\partial r} dr = \dot{f}(r) dr$.

The determinant of the induced metric and its square rood will be
\eq{1h}{
    h = \left( \frac{L}{f(r)} \right) ^4 r^2 ( 1 + \dot{f}(r)^2 ) \ , \
    \sqrt{h} = \left( \frac{L}{f(r)} \right) ^2 r \sqrt{ 1 + \dot{f}(r)^2 } \ .
}

The minimal value of the integral over the polar coordinates of the square rood of the induced metric will correspond to the area of $\gamma_{\calA}$. So, by the Ryu-Takayanagi formula, the entanglement entropy related to the region $\calA$ will be
\eq{1EEA}{
    \arraycolsep=1.4pt\def\arraystretch{2}
    \begin{array}{c}
        S_{\calA} = \frac{1}{4G} \text{min} \int_{\gamma_{\calA}} \sqrt{h} dx^\rho = \\
        = \frac{1}{4G} \text{min} \int_0^{2\pi} d\theta \int_0^R dr \left( \frac{L}{f(r)} \right) ^2 r \sqrt{ 1 + \dot{f}(r)^2 } \\
        = \frac{\pi L^2}{2G} \text{min} \int_0^R dr \frac{r}{f(r)^2} \sqrt{ 1 + \dot{f}(r)^2 } \ .
    \end{array}
}

The interior of this final integral looks like some type of Lagrangian $\calL [r,f(r),\dot{f}(r)]$. Thus, the Euler-Lagrange equation can be aplied to find relations to find the extreme of this functional:
\eq{1EL}{
    \arraycolsep=1.4pt\def\arraystretch{2}
    \begin{array}{c}
        \frac{\partial \calL}{\partial f} - \frac{d}{dr} \left[ \frac{\partial \calL}{\partial \dot{f}} \right] = 0 \\
        \rightarrow \left( 1+\dot{f}^2 \right) \left( -2r-f\dot{f}-rf\ddot{f} \right) + rf\dot{f}^2\ddot{f} = 0
    \end{array}
}

It is proven that $f(r) = \sqrt{R^2 - r^2}$ is solution of the previous solution and correspond to the function that minimise the functional of the entanglement entropy. Then:
\eq{1sol}{
    \arraycolsep=1.4pt\def\arraystretch{2}
    \begin{array}{c}
    S_{\calA} = \frac{\pi L^2}{2G} \int_0^R dr \frac{r}{\left(R^2-r^2\right)^{3/2}} = \\
    = \frac{\pi L^2}{2G} \left. \frac{R}{\sqrt{R^2-r^r}} \right|_{r=R} - \frac{\pi L^2}{2G} = \frac{\pi L^2}{2G} \frac{R}{\delta} - F \ ,
    \end{array}
}
that is equivalent to the general expression for the entanglement entropy in a 3-dimensional QFT with
\[
    \left\{
    \arraycolsep=1.4pt\def\arraystretch{2}
    \begin{array}{lr}
        c_1 = \frac{\pi L^2}{2G} \\
        \delta = \lim_{r \to R} \sqrt{R^2-r^2} \\
        s_{\text{non-loc}} = - \frac{i \pi L^2}{2G} & .
    \end{array}
    \right.
\]






\printbibliography

\end{document}