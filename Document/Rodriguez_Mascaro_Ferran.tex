\documentclass[lettersize,journal]{IEEEtran}

\usepackage[utf8]{inputenc}
\usepackage{lmodern}
\usepackage[T1]{fontenc}
\usepackage{mathtools}
\usepackage{float}
\usepackage{csvsimple}
\providecommand{\abs}[1]{\lvert#1\rvert}
\usepackage{braket}
\providecommand{\eq}[2]{
    \begin{equation}
        #2
    \label{eq:#1}
    \end{equation}
}
\usepackage{amsmath}
\DeclareMathOperator{\calA}{\mathcal{A}}
\DeclareMathOperator{\calH}{\mathcal{H}}
\DeclareMathOperator{\tr}{tr}
\usepackage{fancyhdr}
\usepackage{authblk}
\usepackage{abstract}
\usepackage{wrapfig}
\usepackage{graphicx}

\usepackage[backend=biber,style=phys]{biblatex}
\addbibresource{TFG.bib}

% \usepackage{multirow}
% \usepackage[table,xcdraw]{xcolor}

% \usepackage{graphicx}
% \usepackage{caption}
% \usepackage{subcaption}

% \usepackage[a4paper]{geometry}
% \geometry{top=3cm, bottom=3.3cm, left=3cm, right=3cm}


\title{Entanglement Entropy and Holography}
\author{Ferran Rodríguez Mascaró}
\date{}

\begin{document}


\maketitle{}


% \twocolumn[
%     \begin{@twocolumnfalse}
%         \maketitle{}
%         \begin{abstract}
%         ...
%         \end{abstract}
%     \end{@twocolumnfalse}
% ]


\section{Introduction}


\subsection{Anti-de Sitter space-times}

An anti-de Sitter (AdS) space-time is a maximally symmetric spacetime with negative curvature, solution to Einstein's equations with a negative cosmological constant.

% A maximally symmetric spacetime with negative curvature, solution to Einstein's equations with a negative cosmological constant, is called an anti-de Sitter (AdS) spacetime.

% Its Riemann tensor is expressed as
% \begin{equation}
%     R_{abcd} = - \frac{1}{L^2} ( g_{ac} g_{bd} - g_{ad} g_{bc} ) ,
% \label{eq:AdS_Riemann}
% \end{equation}
% One can find that for these type of space-times to be solution of Einstein equations, the cosmological constant must be
% \begin{equation}
%     \Lambda = - \frac{(D-1)(D-2)}{2L^2} .
% \label{eq:AdS_cosmo-const}
% \end{equation}

One can obtain the metric of the half-space of an AdS spacetime of $D=d+1$ dimensions using the coordinate system of the Pointcaré patch as
\eq{AdS_PP-metric}{
    ds_{AdS_D}^2 = \frac{1}{z^2} \left( -dt^2 + dz^2 + \sum_{i=1}^{d-1} dx_i^2 \right) \ ,
}
with the time and space-related dimensions $t , x_i \in (-\infty,+\infty)$ and an extra dimension $z \in (0,+\infty)$ \cite{kaplan_lectures_nodate}.

Fixing the coordinate $z$, one creates $d$-dimensional Minkowski space-time surfaces 'weighted' by the factor $\frac{1}{z^2}$.

\begin{figure}[h!]
    \centering
    \includegraphics[scale=0.25]{../Imatges/Captura_Superficies_z.png}
\caption{Representation of the different Minkowski space-time layers along the $z$ coordinate inside an AdS space-time.}
\label{fig:AdS_z-surfaces}
\end{figure}

At constant time, \cite{} this metric forms hyperbolic spaces of negative curvature, conformally equivalent to Minkowski space-times at $z \to 0$. The conformal infinity of AdS is timelike, thus one needs boundary conditions to determine the future evolution uniquely.
% \footnotetext{A conformal or angle-preserving function is the one that preserves angles between curves at a certain point, as well as preserving orientation \cite{}.}

Using hyper-polar coordinates one can obtain a different expression for the metric which covers the entire space:
\eq{AdS_hyper-polar-metric}{
    ds_{AdS_D}^2 = \left [ - \left ( 1 + \frac{r^2}{L^2} \right ) dt^2 + \frac{dr^2}{\left ( 1+ \frac{r^2}{L^2} \right )} + r^2 d \Omega_{D-2}^2 \right ] \ ,
}
being $L$ ($k^2=1/L^2$) the so called anti-de Sitter radius.

\begin{figure}[h!]
    \centering
    \includegraphics[scale=0.5]{../Imatges/Wikipedia_Half-space_Cilindric.png}
\caption{Representation of the half-space region of an AdS space-time and its boundary.}
\label{fig:AdS_cylindrical}
\end{figure}

In Figure \ref{fig:AdS_cylindrical} it is represented the metric of Equation \ref{eq:AdS_hyper-polar-metric}. The whole cylinder corresponds to an AdS space-time, being the lateral surface the conformal boundary (where $z \to 0$, or $r \to \infty$). The marked region corresponds to the one covered by the half-space coordinates, flanked by the conformal boundary and two lightlike geodesic hyperplanes \cite{}.


\subsection{Conformal Field Theories}

A conformal field theory (CFT) is a quantum field theory that is invariant under transformations that locally preserve angles.


\subsection{The Holographic Principle}

The covariant entropy bound \cite{bousso_covariant_1999} is conjectured as the representation of the universal law, in a four-dimensional space-time on which Einstein’s equation is satisfied, in which the entropy of a system is bounded by its area. It says that the number of independent degrees of freedom on any light-sheet of a surface cannot exceed a quarter of the area of this surfaces.

This bound implies that the degrees of freedom inside some region grows with the area of the boundary and not as the volume of the region. This behavior leads to the \textit{holographic principle}, wich states that in a quantum gravity theory all the physics phenomena within some volume can be described in terms of a theory on the boundary of the area of the volume, which has less than one degree of freedom per Planck area \cite{t_hooft_dimensional_2009}.


\subsection{AdS/CFT Correspondance}

A class of conformal field theories are equivalently described in certain limits in terms of anti-de Sitter space-times \cite{rangamani_holographic_2017}.

The \textit{AdS/CFT Correspondance}, simply called \textit{holography} in high energy physics, is an equivalence or duality between quantum gravity theories (certain string theories) at asymptotically $D$-dimensional anti-de Sitter space-times and non-gravitational conformal quantum field theories at Minkowski space-times of $D-1$ dimensions \cite{maldacena_large_1999}. It allows us to study different aspects of each of these theories through the other. The so called \textit{holographic dictionary} relates quantities (observables) between the AdS theories and the CFT. % \cite{kaplan_lectures_nodate}


\subsection{Entanglement Entropy}

When two quantum systems enter into temporary physical interaction, they can no longer be described in the same way after a time of mutual influence \cite{schrodinger_discussion_1935}. One can no longer describe neither of those systems independently without losing global information, because the state of each systems know is influenced and correlated by the other system. This is the so called \textit{quantum entanglement}.

Being two quantum systems represented by the corresponding Hillberg spaces $\calH_A$ and $\calH_B$, and an state $\ket{\Psi} \in \calH = \calH_A \otimes \calH_B$, this would be an entangled state if
\eq{entanglement}{
    \ket{\Psi} \neq \ket{\Psi_A} \otimes \ket{\Psi_B} \ \longrightarrow \ \ket{\Psi} = \sum_{i,j} c_{ij} \ket{i}_A \otimes \ket{j}_B \ ,
}
being $\ket{\Psi_{A,B}}$ the possible different substates in which one could separate $\ket{\Psi}$ if it was separable, substates expressed in each orthonormal bases $\{ \ket{k}_{A,B} \}$ of $\calH_{A,B}$ \cite{}.

The \textit{entanglement entropy} is a measure of the degree of quantum entanglement between the two subsystems composing a full quantum system \cite{nishioka_entanglement_2018}. It is defined by the von
Neumann entropy of the reduced density matrix $\rho_A$ of one of the subsystems as
\eq{entanglement-entropy}{
    S_{EE}(A) = - \tr_A ( \rho_A \log \rho_A ) \ ,
}
being $\rho_A = \tr_B \ket{\Psi}\bra{\Psi}$. If $\rho_A$ is diagonalized ($\rho_A = \sum_i \lambda_i \ket{i}\bra{i}$), then the entanglement entropy would take the simplified form $S_{EE} = - \sum_i \lambda_i \log \lambda_i$.

If there is no entanglement between both subsystems, the entanglement entropy is null ($S_{EE} = 0$).


\subsection{Entanglement entropy in AdS/CFT}

In a quantum field placed in a Minkowski space-time, at a given time, every point on space is entangled with the points surrounding it \cite{nishioka_entanglement_2018}. Therefore, the entanglement entropy between a subsytem and the rest of the space will be dominated by the correlations between both sides of the boundary that isolates the subsystem (Figure \ref{fig:EE_Minkowski-boundary}).

% \begin{wrapfigure}{r}{0.5\textwidth}
%     \centering
%     \includegraphics[width=0.48\textwidth]{../Imatges/Captura_EE_Frontera.png}
% \caption{Bipartion of a system in two complementary regions A and B.}
% \label{fig:EE_Minkowski-boundary}
% \end{wrapfigure}

% \begin{figure}[h!]
%     \centering
%     \includegraphics[scale=0.2]{../Imatges/Captura_EE_Frontera.png}
% \caption{Bipartion of a system in two complementary regions A and B.}
% \label{fig:EE_Minkowski-boundary}
% \end{figure}

\begin{figure}[h!]
    \centering
    \includegraphics[scale=0.25]{../Imatges/EE_AdS-CFT.png}
\caption{Region $\calA$ (dark blue) and its boundary $\partial \calA$ inside a $z=\delta$ AdS slide (grey) and its respective $\gamma_{\calA}$ (light blue) inside the AdS space-time.}
\label{fig:EE_AdS/CFT}
\end{figure}

In an ($d+1$)-dimensional AdS space-time, being $\calA$ a region of a $d$-dimensional Minkowski space-time slice formed from fixing $z$ as $z=\delta \ll 1$, the entanglement entropy of a $d$-dimensional CFT on this Minkowski space-time will be expressed by the so called Ryu-Takayanagi formula
\eq{EE_RT}{
    S_{\calA} = \frac{ \text{Area}(\gamma_{\calA}) }{ 4 G_{d+1} } \ ,
}
\cite{ryu_holographic_2008} where $\gamma_{\calA}$ is the surface of minimal area on the whole AdS space-time connected to the ($d-1$)-dimensional boundary $\partial \calA$ of the region $\calA$, and $G_{d+1}$ is the ($d+1$)-dimensional Newton constant (represented in Figure \ref{fig:EE_AdS/CFT}).

The area of $\gamma_{\calA}$ is obtained by
\eq{EE_RT-area}{
    \text{Area}(\gamma_{\calA}) = \int_{\gamma_{\calA}} \sqrt{h} \ d^{d}y \ ,
}
where $y$ are the $d$ coordinates that represent surface $\gamma_{\calA}$ and $h$ is the determinant of the metric $h_{ij} = \frac{\partial x^\mu}{\partial y^i} \frac{\partial x^\nu}{\partial y^j} g_{\mu\nu}$ induced on the surface by the surrounding space-time.

The Ryu-Takayanagi formula is valid for generic systems, and gives a flavour of how the geometry of space-time can emerge from mere quantum information. As a curiosity, the Ryu-Takayanagi formula in the case of a thermalized system of particles in an AdS space-time derives to the Bekenstein-Hawking formula \cite{bekenstein_black_1973} for the entropy a black hole:
\eq{BH}{
    S_{BH} = \frac{ A_H }{ 4 G } \ ,
}
that says that the entropy related to a black hole only depends on the area $A_H$ of its event horizon.

Solving the Ryu-Takayanagi formula for a ($d+1$)-dimensional anti-de Sitter space-time one obtains the expected general expression of the entanglement entropy for a $d$-dimensional quantum field theory:
\eq{EE}{
    S_{QFT_d} = \sum_{i=0}^{d/2-1} c_i \left( \frac{R}{\delta} \right) ^{d-2(i+1)} + a \log \left( \frac{R}{\delta} \right) - F \ ,
}
\cite{nishioka_entanglement_2018} in which $R$ is the characteristic length of the region studied, $c_i$ are coefficients that can be dependent on $\delta$, the logarithmic component only appears for even $d$, and $F$ is a function related to the aspect of the region $\calA$.


\printbibliography

\end{document}